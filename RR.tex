\documentclass[]{article}
\usepackage{geometry}
\usepackage{hyperref}
\begin{document}

\begin{center}
\Large \textbf{Veronica's Research Roadmap}
\end{center}

My target phenomena is the well-attested reduction to partner-specific abbreviated conventional names that occurs in iterated reference games for initially difficult to name targets. I propose a set of chapters exploring reduction phenomena across varying contexts as well as looking more closely at the dynamics and whether the pattern of partner-specificity and reduction enhances comprehension. 

\section{Interaction structure constrains the emergence of conventions in group communication}
We ask which properties of the group’s interaction structure facilitate
successful communication. We used a repeated reference game paradigm in which directors
instructed between one and five matchers to choose specific targets out of a set of abstract
figures. Across 313 games (\textsc{n} = 1,319 participants), we manipulated several key constraints on
the group’s interaction, including the amount of feedback that matchers could give to directors
and the availability of peer interaction between matchers. Across groups of different sizes and
interaction constraints, describers produced increasingly efficient utterances and matchers made
increasingly accurate selections. Critically, however, we found that smaller groups and groups
with less-constrained interaction structures (“thick channels”) showed stronger convergence
to group-specific conventions than large groups with constrained interaction structures (“thin
channels”), which struggled with convention formation. 

\textbf{Status}: \href{https://osf.io/preprints/psyarxiv/a3wfy}{Preprint}


\section{Dynamics of language use in reduction}
Can we come up with a useful classification scheme / something to get traction on how reduction/convention formation actually happens. Current idea is to break utterances into their conceptual ``chunks'' and look at how many and what types of chunks are used when, and whether this gives useful traction on the dynamics of reduction / convention-formation? 

\textbf{Status:} Attempts were made this fall (\href{https://rpubs.com/vboyce/tggpt1}{progress thus far}), and back-burnered while I think about connections with other metaphorical language use. Have a 224N group working on roughly this which might give me ideas for technical tools that work. 
Current pitfalls are figuring out how to frame/analyze is a way that generalized beyond humanoid-tangrams, and how to automate well (GPT was doing worse than regex at some category labels). Plan to resume in spring quarter. 

\section{Comprehension side of reducing utterances}
There's a couple related threads here. One is testing the partner-specificity and efficiency claims that are made by seeing whether naive listeners can understand utterances from different time points in a reference game (possibly x game condition). 

Secondly, I'm curious for the multi-part descriptions, how early the target is identifiable using incremental methods (web-eye-tracking, or a self-paced-reading/gating paradigm). 

I think there's a rich area to explore here, but it's not clear how much I'll get to. I think the right start point for a pilot/experiment 1 is looking at accuracy of naive listeners on final round utterances (from varying games where the in-game listeners were successful). 

\textbf{Status:} Not started. Hoping to set up and pilot this winter, run more in spring, and eventually shift to incremental paradigm. 

\section{4-5 year old children can successfully communicate using ad-hoc referential expressions}
Iterated reference games are commonly used with adults, but the consensus has been that tracking ad-hoc references in this way is too difficult for young children (in part due to notable failures in the literature; Glucksberg et al., 1966). We revisit this question, using a simplified method adapted from Leung et al. 2020. 20 pairs of children played a matching game on tablets. Children use a variety of different referential expressions to successfully convey the target to their partner. Children were 84\% accurate at selecting the target image. Children generally produced short descriptions, and the length of descriptions tended to slightly increase over time. Children's descriptions for an image are more similar to their partner's descriptions than the descriptions of children in other games, indicating some level of partner specificity. This preliminary study suggests that (contra Glucksberg et al.) 4-5 year old children can describe and select novel figures with each other at above chance accuracy.

\textbf{Status:} Expt 1 (20 dyads) completed (\href{https://github.com/vboyce/kid-tangrams/blob/main/write-ups/CAMP6/abstract.pdf}{CAMP abstract}). Expt 2 is a replication of expt 1 with 30 dyads and minor engineering tweaks is piloting now, hopefully collecting data winter-spring (and into summer if needed). Expt 3 (some tbd minor extension) expected over summer. Aim to write up the 3 expts for journal/cogsci summer-fall. 


\section*{Not appearing in thesis}

[Lest you think I've done very little in grad school or since the last committee meeting, here's other done,  on-going, and planned work that won't be in the thesis.]\\

\noindent\textbf{Probably abandoned: Coordination and convention formation across settings} 

There were previous attempts to look at the generality of reduction/convention formation in strategic games. Results from a 3 player coordination/negotation game are written up in  \href{https://osf.io/preprints/psyarxiv/tfb3d}{joint workshop paper} (game results) and \href{https://escholarship.org/uc/item/8dq8c2s6}{cogsci} (language analyses). I did some experiments on reference in more traditional game-theory games (\href{https://docs.google.com/presentation/d/1CzJ6fl6PEJ6NepUwjKKINLlVa80AAEMAQj4s5idrwIA/edit#slide=id.p}{slides}). \\


\noindent\textbf{Meta (Analysis|Science): the side project}
\begin{itemize}
	\item (joint work) \href{https://osf.io/preprints/psyarxiv/qc6fw}{Conducting developmental research online vs. in-person: A meta-analysis} (preprint)
	\item \href{https://royalsocietypublishing.org/doi/full/10.1098/rsos.231240}{Eleven years of student replication projects provide evidence on the correlates of replicability in psychology} (published at Royal Society Open Science)
	\item  Ongoing 251 ``Rescue'' project: Data collected, analysis and writing ongoing
	\item Possible future project: formal modeling expected replication outcomes in terms of heterogeneity and QRPs
\end{itemize}
%\section{Coordination and convention formation across settings}
%A different dimension along which to explore the generality of the reduction result is in strategic games. 
%
%\textbf{Status:} We looked at this in the context of a coordination/negotiation game in the AA-flowers paper (NeurIPS 21; language analysis submitted for cogsci 23), but it was hard to interpret since the condition manipulation was very weak. To address that, we tried using game theory games (where strategy, at least in one-shot, is better understood) in particular to look for convention formation around \textit{strategies}; I ran a few preliminary experiments with some suggestive patterns, but participants in general don't act like game theory models. The goal here would be to get a situation in which conventions form on some identifiably ``abstract'' measure, but that might require careful game design.
%
%\textbf{Next steps:}  Return to game theory, potentially looking at ways to either encourage abstractions in communication or make strategies more complicated. 
%
%\textbf{Potential pitfalls:} Failure to get experimental designs that are appropriately calibrated to encourage language use and have condition differences. 
%For the sake of completeness, here are some other ideas. 

%I could run more iterated reference game experiments on adults; two obvious directions are going to still larger groups, or looking for how schemas for how to describe the images might transfer to new images, by switching out some or all of the images midway through, but maintaining the same group composition. Could be interesting, but I mostly think I should work on the language data I have and do some targeted solo experiments first. 

%Meta-analyses are always cool, but I don't think the iterated reference game literature is amenable as there isn't a unified effect size that would make sense, and the study-moderator ratio is not favorable. I think the closest I can get is a mega-analysis approach using data from studies where it's available, which I would do if the language analyses work out.

%\section*{Not included in the thesis: Meta-analysis and meta-science}
%It should be noted that there's some grad school work that won't be going in because it's meta-analysis of replications. 



\end{document}
