
\documentclass[]{article}
\usepackage[]{geometry}
\usepackage{biblatex}
%opening
\title{ARRR synopsis}
\author{Veronica Boyce}
\addbibresource{sources.bib}
\begin{document}
	
	\maketitle
	

One preliminary question is one of scoping: are we talking about communication or language or face-to-face oral language? Different papers explicitly or implicitly consider different of these domains. Here, I will primarily focus on linguistic communication (writ large) with the understanding that some of the pressures behind communication and convention-formation occuring in other modalities are also informative. However, the backdrop of linguistic habits and the extensive expectations it creates make languistic communication a particularly rich domain for pragmatics and convention formation. 

\subsection{Efficiency}
One unifying framework gaining traction in psycholinguistics is efficiency, the idea that language and language use is under pressure to support efficient communication by maximizing the ratio of relevant information transmitted to effort. Efficiency is thought to arise from trade-offs between communicative expressivity and learnability or easy of production. Many features of language distributions are argued to be much closer to the Pareto frontier than would be expected under a chance model: these features include the distribution of word frequencies, the lexical partitioning of subdomains such as color and kinship terms, and syntactic features such as harmonic word order or dependency length. 

What counts as efficient? The posited pressures all apply to language use, imposing a joint constraint on the entire communicative process to minimize the total time and effort involved in going from an idea in one person's head to a sufficiently close idea in another person's head. Thus shorter utterances (as measured in syllables or clock-time) are not always efficient if they take longer to produce or parse. 

It would seem that ``redundant'' color adjective use and other forms of so-called ``over-informative'' language use run counter to the idea of efficiency. Studying the use of ``redundant'' forms is interesting, but defining what is minimally informative requires commitment to a full semantic-pragmatic system. What is informative is an empirical matter of determining how interlocuters respond to alternatives. Determining what is efficient requires not just analyses the words said and their alternatives, but also production and comprehension time, which may be highly contingent on contextual factors and conversational history. 

Efficiency is very hard to cache out in specific predictions for how people will use language because of the many time scales: what's efficient for an utterance or conversation in isolation may not be efficient when considered over an entire life of language use. Thus, the efficiency framework is dependent on linking assumptions, and an efficiency approach could be seen as being about which linking assumptions are needed and how parsimonious they are (or aren't). 


\subsection{Communication}
The communication and conversation literature, especially with regard to referring expressions, has provided useful descriptive work, but is primarily made up of verbal theories that  hypothesize ``mechanisms'' or principles, which either contradict the data or do not make risky predictions. Many of these theories take a deterministic approach, not directly dealing with the inherent uncertainty and probabilistic nature of interacting with other minds. Many of the terms are also used inconsistently. 

One broad approach to conversation assumes that humans have some sort of mental representation of other humans as agents who can have different mental states. This still leaves a lot of scope for how these representations are implemented, how information gets added or modified, what exactly is tracked, and what processes refer to these representations, rather than use heuristics. 

To reference shared knowledge, many use the term ``common ground''. In some cases, it is used to mean roughly ``things you think another person will understand and won't be surprised if you reference'' which is a useful pre-theoretic idea. However, others use ``common ground'' in a theoretically-loaded way where there is infinite recursion and many things must be introduced into common ground via accommodation. 

An alternative theory to mentalizing approaches called the interactive alignment theory attempts to explain how people can successfully collaborate on reference tasks without reasoning about each other's mental states. This work claims that the alignment occurs via ``priming'' is ``resource-free and automatic'', without providing a further explanation of what this means or how this is working on the level of processing, memory, and production. 

One key phenomenon in reference games is the claim of ``partner specificity''. This is used both in cases when there are actually multiple partners who are given different descriptions and when each person only has one partner, but pairs evolve in language use time in different ways. The empirical evidence seems to point to people doing ``partial pooling'' over partners \cite{hawkins2021}. That is, there is some variation in how people interact that is specific to the partner, but people are also doing some generalization over partners \cite{yoon2014}. 

Related to partner-specificity is the idea of ``audience design'' where speakers seem to be sensitive to the knowledge state of their listener and trying to make the task easier for their listener in what they say and in what order. Confusingly, ``audience design'' sometimes implies intention on the part of the speaker and sometimes is used where the utterance was likely shaped by speaker ease, with a side effect of listener benefit \cite{horton1996, rogers2013}. The study of audience design has raised on important topic of inquiry, namely, ``how do interlocuters split the communicative load with one another?''

Another key observation from repeated reference games is that partners form shared conventions about how to refer to initially-ambiguous targets. The idea of convention formation or a ``conceptual pact'' is ambiguous between different levels of analysis: it could be a pact to refer to a figure as ``ballerina''; it could be thinking of the figure as a ballet dancer with a tutu (manifesting in descriptions that may not overlap lexically, such as ``ballerina'' and ``dancing in a tutu''); or it could be a general principle to think of the figures as people and describe them in terms of postures. Expectations on multiple levels can coexist. One observation is that conventions seem to be partner and context specific: changes in the speaker, audience members, or changes in the context can all license a new description. 

The study of reference games has uncovered interesting descriptive phenomena but does not provide algorithmic explanations or clear demarcations for when they should occur. 

\subsection{RSA}

RSA is an information-theoretic, computational framework for making quantitative predictions about pragmatic inferences in context \cite{goodman2016, frank2012a}. The basic idea of the Rational Speech Acts (RSA) family of models is to picture two interlocuters recursively reasoning about how the other would produce or interpret utterances, grounding out in a listener (or speaker) who behaves in a pre-specified ``literal'' way. Computational frameworks such as RSA provide a way to factor together different trade offs and determine their relative weights in a model. 


Perhaps the largest challenge to RSA models is the question of how to ground out the models in a ``literal'' listener or speaker. For the most part, RSA is tested in toy domains where the set of possible utterances are small and it is possible to enumerate a set of meanings. In less toy domains, there is not a satisfactory answer: some situations can be handled by empirically measuring likelihoods in an exhaustive ways, but this holistic approach is not compatible with incremental RSA or larger sets of utterances that require compositionality to be defined. In order to extend this model towards more realistic and open-ended scenarios, an important question to grapple with is what form of meaning (even at a computational level) will appropriately support pragmatic reasoning. 

One attempt to more directly connect RSA models with convention formation is CHAI, a framework to bridge different levels of convention formation \cite{hawkins2021}. 

\subsection{Psycholinguistic}

Whereas RSA is a computation-level model, psycholinguistics imposes constraints on the algorithmic level of linguistic communication. The big issue is that determining what the constraints are requires understanding both production and parsing, and understanding the time course of production is particularly difficult to study. 

One major point of disagreement in the psycholingistics of referring expressions is whether production and comprehension are initially ``ego-centric'' or whether non-linguistic information, such as the perspective of the interlocuter, is exerting an top down influence from the beginning. This narrow issue raises larger questions about the relative influences of top-down and bottom-up factors in influencing language processing and production.

Production poses a possible deviation from the idealized RSA models in that production requires the retrieval or generation of a potential good enough utterance in the first place. The difficulties of utterance planning may cause deviations from what information-theory would predict would be efficient, based on production biases such as easy first, plan reuse, and reduce interference \cite{macdonald2013}. 

\subsection{Where next?}

It's fine for theories to have different scopes and be at different levels of analysis, but they should be clear about this, and be consistent with theories that cover different things (i.e. no discontinuities in the language to non-language signals gradient, no wacky assumptions about memory).

Both efficiency and RSA seem like promising (and compatible) frameworks. It is unclear how well they will explain reference phenomena, but the question of what linking assumptions and assumptions are needed to unify reference phenomena such as partner-specificity and reduction with the framework is the interesting one. For RSA, one particular questions is what sort of RSA model and semantic system could support the type of model that would be needed. 

While the theories from communication are unsatisfying, the highlight a number of interesting phenomena and the nuances that will need to be explained. Many of these may be easier to accomodate in a probabilistic approach. 

YOU ARE HERE SAY SOMETHING ABOUT PSYCHOLING

\subsection{Open questions}
* From an empirical perspective, how efficient are people in reference games? 

* What is a theory that can explain reduction including making predictions about the rate at which reduction occurs and what words are dropped? Robert's toy models are the only thing that tries to explain how longer initial utterances could then make shorter ones optimal, but the toy nature of the model leaves a lot of room for explaining the trajectory. 

\cite{leung2023} points out that forming a convention can be thought of as preceding in two stages: first some referential expression must succeed in communicating the target, and only then can that expression turn into a more reduced form.


* Groups??? One question is how various theories would scale up to make predictions about even small group interactions. What would a multi-person RSA look like? There some group verbal theories (mostly Yoon in ref game land), but ideas of CG/conceptual pact or alignment don't cache out in clear predictions or say it's just dyads which clearly isn't the whole story. 

\end{document}